\documentclass[a4paper, 12pt]{article} % Font size (can be 10pt, 11pt or 12pt) and paper size (remove a4paper for US letter paper)
\usepackage{amsmath,amsfonts,bm}
\usepackage{hyperref}
\usepackage{amsthm,epigraph} 
\usepackage{amssymb}
\usepackage{framed,mdframed}
\usepackage{graphicx,color} 
\usepackage{mathrsfs,xcolor} 
\usepackage[all]{xy}
\usepackage{fancybox} 
\usepackage{xeCJK}
\usepackage{pstricks-add}
\pagestyle{empty}
\newtheorem*{adtheorem}{定理}
\setCJKmainfont[BoldFont=FZYaoTi,ItalicFont=FZYaoTi]{FZYaoTi}
\definecolor{shadecolor}{rgb}{1.0,0.9,0.9} %背景色为浅红色
\newenvironment{theorem}
{\bigskip\begin{mdframed}[backgroundcolor=gray!40,rightline=false,leftline=false,topline=false,bottomline=false]\begin{adtheorem}}
    {\end{adtheorem}\end{mdframed}\bigskip}
\newtheorem*{bdtheorem}{定义}
\newenvironment{definition}
{\bigskip\begin{mdframed}[backgroundcolor=gray!40,rightline=false,leftline=false,topline=false,bottomline=false]\begin{bdtheorem}}
    {\end{bdtheorem}\end{mdframed}\bigskip}
\newtheorem*{cdtheorem}{习题}
\newenvironment{exercise}
{\bigskip\begin{mdframed}[backgroundcolor=gray!40,rightline=false,leftline=false,topline=false,bottomline=false]\begin{cdtheorem}}
    {\end{cdtheorem}\end{mdframed}\bigskip}
\newtheorem*{ddtheorem}{注}
\newenvironment{remark}
{\bigskip\begin{mdframed}[backgroundcolor=gray!40,rightline=false,leftline=false,topline=false,bottomline=false]\begin{ddtheorem}}
    {\end{ddtheorem}\end{mdframed}\bigskip}
\newtheorem*{edtheorem}{引理}
\newenvironment{lemma}
{\bigskip\begin{mdframed}[backgroundcolor=gray!40,rightline=false,leftline=false,topline=false,bottomline=false]\begin{edtheorem}}
    {\end{edtheorem}\end{mdframed}\bigskip}
\newtheorem*{pdtheorem}{例}
\newenvironment{example}
{\bigskip\begin{mdframed}[backgroundcolor=gray!40,rightline=false,leftline=false,topline=false,bottomline=false]\begin{pdtheorem}}
    {\end{pdtheorem}\end{mdframed}\bigskip}

\usepackage[protrusion=true,expansion=true]{microtype} % Better typography
\usepackage{wrapfig} % Allows in-line images
\usepackage{mathpazo} % Use the Palatino font
\usepackage[T1]{fontenc} % Required for accented characters
\linespread{1.05} % Change line spacing here, Palatino benefits from a slight increase by default

\makeatletter
\renewcommand\@biblabel[1]{\textbf{#1.}} % Change the square brackets for each bibliography item from '[1]' to '1.'
\renewcommand{\@listI}{\itemsep=0pt} % Reduce the space between items in the itemize and enumerate environments and the bibliography

\renewcommand{\maketitle}{ % Customize the title - do not edit title
  % and author name here, see the TITLE block
  % below
  \renewcommand\refname{参考文献}
  \newcommand{\D}{\displaystyle}\newcommand{\ri}{\Rightarrow}
  \newcommand{\ds}{\displaystyle} \renewcommand{\ni}{\noindent}
  \newcommand{\pa}{\partial} \newcommand{\Om}{\Omega}
  \newcommand{\om}{\omega} \newcommand{\sik}{\sum_{i=1}^k}
  \newcommand{\vov}{\Vert\omega\Vert} \newcommand{\Umy}{U_{\mu_i,y^i}}
  \newcommand{\lamns}{\lambda_n^{^{\scriptstyle\sigma}}}
  \newcommand{\chiomn}{\chi_{_{\Omega_n}}}
  \newcommand{\ullim}{\underline{\lim}} \newcommand{\bsy}{\boldsymbol}
  \newcommand{\mvb}{\mathversion{bold}} \newcommand{\la}{\lambda}
  \newcommand{\La}{\Lambda} \newcommand{\va}{\varepsilon}
  \newcommand{\be}{\beta} \newcommand{\al}{\alpha}
  \newcommand{\dis}{\displaystyle} \newcommand{\R}{{\mathbb R}}
  \newcommand{\N}{{\mathbb N}} \newcommand{\cF}{{\mathcal F}}
  \newcommand{\gB}{{\mathfrak B}} \newcommand{\eps}{\epsilon}
  \begin{flushright} % Right align
    {\LARGE\@title} % Increase the font size of the title
    
    \vspace{50pt} % Some vertical space between the title and author name
    
    {\large\@author} % Author name
    \\\@date % Date
    
    \vspace{40pt} % Some vertical space between the author block and abstract
  \end{flushright}
}

% ----------------------------------------------------------------------------------------
%	TITLE
% ----------------------------------------------------------------------------------------
\begin{document}
\title{\textbf{过圆外一点做两直线与圆相交形成的四边形面积最大值,未完成}} 
% \setlength\epigraphwidth{0.7\linewidth}
\author{\small{叶卢庆}\\{\small{杭州师范大学理学院}}\\{\small{Email:h5411167@gmail.com}}} % Institution
\renewcommand{\today}{\number\year. \number\month. \number\day}
\date{\today} % Date

% ----------------------------------------------------------------------------------------


\maketitle % Print the title section

% ----------------------------------------------------------------------------------------
%	ABSTRACT AND KEYWORDS
% ----------------------------------------------------------------------------------------

% \renewcommand{\abstractname}{摘要} % Uncomment to change the name of the abstract to something else

% \begin{abstract}

% \end{abstract}

% \hspace*{3,6mm}\textit{关键词:}  % Keywords

% \vspace{30pt} % Some vertical space between the abstract and first section

% ----------------------------------------------------------------------------------------
%	ESSAY BODY
% ----------------------------------------------------------------------------------------
如图,已知一定圆,过圆外一定点 $P$ 作直线 $PD,PC$ 分别与圆相交于
$A,B,C,D$ 四点.求四边形 $ABCD$ 的面积何时最大,以及最大值为多少.\\
\newrgbcolor{xdxdff}{0.49 0.49 1}
\psset{xunit=1.0cm,yunit=1.0cm,algebraic=true,dotstyle=o,dotsize=3pt 0,linewidth=0.8pt,arrowsize=3pt 2,arrowinset=0.25}
\begin{pspicture*}(-3.9,-6.06)(16.42,6.12)
\pscircle(1.46,0.26){4.83}
\psline(10.84,3.1)(-2.28,3.32)
\psline(10.84,3.1)(-0.6,-4.11)
\psline(-2.28,3.32)(-0.6,-4.11)
\psline(5.34,3.15)(6.29,0.23)
\begin{scriptsize}
\psdots[dotstyle=*,linecolor=blue](1.46,0.26)
\rput[bl](1.54,0.38){\blue{$O$}}
\psdots[dotstyle=*,linecolor=blue](10.84,3.1)
\rput[bl](10.92,3.22){\blue{$P$}}
\psdots[dotstyle=*,linecolor=xdxdff](-2.28,3.32)
\rput[bl](-2.2,3.44){\xdxdff{$D$}}
\psdots[dotstyle=*,linecolor=xdxdff](-0.6,-4.11)
\rput[bl](-0.52,-4){\xdxdff{$C$}}
\psdots[dotstyle=*,linecolor=xdxdff](5.34,3.15)
\rput[bl](5.42,3.26){\xdxdff{$A$}}
\psdots[dotstyle=*,linecolor=darkgray](6.29,0.23)
\rput[bl](6.38,0.36){\darkgray{$B$}}
\end{scriptsize}
\end{pspicture*}
\begin{proof}[失败的尝试1]
不妨设圆的半径为 $r$,点 $P$ 与圆心 $O$ 的距离为 $p$.则根据圆幂定理,可
得
\begin{equation}
  \label{eq:1}
  |PA||PD|=p^2-r^2.
\end{equation}
\begin{equation}
  \label{eq:2}
  |PB||PC|=p^2-r^2.
\end{equation}
由 \eqref{eq:1} 和 \eqref{eq:2} 可得
\begin{equation}
  \label{eq:3}
  \frac{|PA|}{|PC|}=\frac{|PB|}{|PD|}=k.
\end{equation}
因此,三角形 $PAB$ 与三角形 $PCD$ 相似.且
\begin{equation}
  \label{eq:4}
  \frac{S_{PAB}}{S_{PCD}}=k^2.
\end{equation}
将 \eqref{eq:3} 代入 \eqref{eq:1},可得
\begin{equation}
  \label{eq:5}
  \frac{|PA||PB|}{p^2-r^2}=k,
\end{equation}
我们知道,
\begin{equation}
  \label{eq:6}
  S_{ABCD}=S_{PCD}-S_{PAB}=S_{PAB}(\frac{1}{k^2}-1).
\end{equation}
因此将 \eqref{eq:5} 代入 \eqref{eq:6} 可得
\begin{equation}
  \label{eq:7}
  S_{ABCD}=S_{PAB}\frac{(p^2-r^2)^2-|PA|^2|PB|^2}{|PA|^2|PB|^2}.
\end{equation}
我们知道,
\begin{equation}
  \label{eq:8}
  S_{PAB}=\frac{1}{2}|PA||PB|\sin\angle APB,
\end{equation}
我们发现,将 \eqref{eq:8} 代入 \eqref{eq:7} 后并不能让四边形 $ABCD$ 的
面积表达式简化.我们只好另寻它路.
\end{proof}
\begin{proof}[第二种解决方案]
如图,\\
\newrgbcolor{xdxdff}{0.49 0.49 1}
\newrgbcolor{qqwuqq}{0 0.39 0}
\psset{xunit=1.0cm,yunit=1.0cm,algebraic=true,dotstyle=o,dotsize=3pt 0,linewidth=0.8pt,arrowsize=3pt 2,arrowinset=0.25}
\begin{pspicture*}(-4,-5.7)(15.52,5.53)
\pscircle(1.46,0.26){4.83}
\psline(10.84,3.1)(-2.28,3.32)
\psline(10.84,3.1)(-0.6,-4.11)
\psline(-2.28,3.32)(-0.6,-4.11)
\psline(5.34,3.15)(6.29,0.23)
\psline(-2.28,3.32)(1.46,0.26)
\psline(1.46,0.26)(-0.6,-4.11)
\psline(1.46,0.26)(5.34,3.15)
\psline(1.46,0.26)(6.29,0.23)
\pscustom[linecolor=qqwuqq,fillcolor=qqwuqq,fillstyle=solid,opacity=0.1]{\parametricplot{2.457041692520875}{4.272990348705159}{0.55*cos(t)+1.46|0.55*sin(t)+0.26}\lineto(1.46,0.26)\closepath}
\pscustom[linecolor=qqwuqq,fillcolor=qqwuqq,fillstyle=solid,opacity=0.1]{\parametricplot{0.6401594336797238}{2.457041692520875}{0.55*cos(t)+1.46|0.55*sin(t)+0.26}\lineto(1.46,0.26)\closepath}
\pscustom[linecolor=qqwuqq,fillcolor=qqwuqq,fillstyle=solid,opacity=0.1]{\parametricplot{-0.005897031940235101}{0.6401594336797239}{0.55*cos(t)+1.46|0.55*sin(t)+0.26}\lineto(1.46,0.26)\closepath}
\begin{scriptsize}
\psdots[dotstyle=*,linecolor=blue](1.46,0.26)
\rput[bl](1.53,0.37){\blue{$O$}}
\psdots[dotstyle=*,linecolor=blue](10.84,3.1)
\rput[bl](10.91,3.21){\blue{$P$}}
\psdots[dotstyle=*,linecolor=xdxdff](-2.28,3.32)
\rput[bl](-2.21,3.43){\xdxdff{$D$}}
\psdots[dotstyle=*,linecolor=xdxdff](-0.6,-4.11)
\rput[bl](-0.51,-4){\xdxdff{$C$}}
\psdots[dotstyle=*,linecolor=xdxdff](5.34,3.15)
\rput[bl](5.4,3.26){\xdxdff{$A$}}
\psdots[dotstyle=*,linecolor=darkgray](6.29,0.23)
\rput[bl](6.36,0.35){\darkgray{$B$}}
\rput[bl](1.07,0.11){\qqwuqq{$\alpha_1$}}
\rput[bl](1.26,0.59){\qqwuqq{$\alpha_2$}}
\rput[bl](1.97,0.46){\qqwuqq{$\alpha_3$}}
\end{scriptsize}
\end{pspicture*}
\end{proof}
% ----------------------------------------------------------------------------------------
%	BIBLIOGRAPHY
% ----------------------------------------------------------------------------------------

\bibliographystyle{unsrt}

\bibliography{sample}

% ----------------------------------------------------------------------------------------
\end{document}